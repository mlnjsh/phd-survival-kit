% =============================================================================
% Response to Reviewers Template
% PhD Survival Kit — Prof. Milan Amrut Joshi
% =============================================================================
%
% USAGE:
%   1. Paste each reviewer comment into the \ReviewerComment{} blocks
%   2. Write your response in the \Response{} blocks
%   3. Describe manuscript changes in the \Change{} blocks
%   4. Compile: pdflatex template.tex
%
% COLOR CODING:
%   - Reviewer comments: BLUE
%   - Your responses: BLACK
%   - Changes made to manuscript: RED
%   - Editor comments: DARK PURPLE
%
% TIPS:
%   - Be thorough but concise
%   - Thank every reviewer, even for harsh criticism
%   - Quote specific page/line/section numbers for changes
%   - Never be defensive; be professional and grateful
%   - Address EVERY point, even if you respectfully disagree
%
% =============================================================================

\documentclass[11pt, a4paper]{article}

% --- Packages ---
\usepackage[margin=1in]{geometry}
\usepackage{xcolor}
\usepackage{hyperref}
\usepackage{enumitem}
\usepackage{parskip}          % Paragraph spacing
\usepackage{titlesec}         % Section formatting
\usepackage{fancyhdr}         % Header/footer
\usepackage{lipsum}           % Placeholder text (remove in final version)

% =============================================================================
% COLOR DEFINITIONS
% =============================================================================
\definecolor{ReviewerBlue}{RGB}{0, 70, 150}
\definecolor{ResponseBlack}{RGB}{0, 0, 0}
\definecolor{ChangeRed}{RGB}{180, 30, 30}
\definecolor{EditorPurple}{RGB}{100, 30, 130}
\definecolor{LightGray}{RGB}{245, 245, 245}
\definecolor{AcceptGreen}{RGB}{30, 130, 60}

% =============================================================================
% CUSTOM COMMANDS
% =============================================================================

% --- Reviewer Comment (blue, italic, indented) ---
\newcommand{\ReviewerComment}[1]{%
    \vspace{0.5em}
    \noindent
    \begin{quote}
        \textcolor{ReviewerBlue}{\textit{#1}}
    \end{quote}
}

% --- Editor Comment (purple, italic) ---
\newcommand{\EditorComment}[1]{%
    \vspace{0.5em}
    \noindent
    \begin{quote}
        \textcolor{EditorPurple}{\textit{#1}}
    \end{quote}
}

% --- Response (black, normal) ---
\newcommand{\Response}[1]{%
    \vspace{0.3em}
    \noindent
    \textbf{Response:} #1
}

% --- Changes made to manuscript (red) ---
\newcommand{\Change}[1]{%
    \vspace{0.3em}
    \noindent
    \textcolor{ChangeRed}{\textbf{Changes made:} #1}
}

% --- Shorthand for point numbering ---
\newcounter{reviewerpoint}
\newcommand{\ReviewerPoint}[1]{%
    \stepcounter{reviewerpoint}
    \vspace{1em}
    \noindent
    \textbf{Point \arabic{reviewerpoint}:}
    \ReviewerComment{#1}
}

% --- Reset counter for each reviewer ---
\newcommand{\ResetPoints}{\setcounter{reviewerpoint}{0}}

% =============================================================================
% HEADER/FOOTER
% =============================================================================
\pagestyle{fancy}
\fancyhf{}
\lhead{\small Response to Reviewers}
\rhead{\small [Manuscript ID: XXXX-XXXX]}
\cfoot{\small Page \thepage}
\renewcommand{\headrulewidth}{0.4pt}

% =============================================================================
% SECTION FORMATTING
% =============================================================================
\titleformat{\section}
    {\Large\bfseries\color{ReviewerBlue}}
    {\thesection}{1em}{}[\vspace{0.3em}\hrule]

\titleformat{\subsection}
    {\large\bfseries}
    {\thesubsection}{1em}{}

% =============================================================================
% DOCUMENT
% =============================================================================
\begin{document}

% --- Title Block ---
\begin{center}
    {\LARGE\bfseries Response to Reviewers}

    \vspace{1em}

    {\large Manuscript: \textbf{``[Your Paper Title]''}}

    \vspace{0.5em}

    {\normalsize Manuscript ID: [XXXX-XXXX]}

    \vspace{0.5em}

    {\normalsize Submitted to: \textit{[Journal Name]}}

    \vspace{0.5em}

    {\normalsize Authors: [Author 1, Author 2, Author 3]}

    \vspace{0.5em}

    {\normalsize Date: \today}
\end{center}

\vspace{1em}
\hrule
\vspace{1em}

% =============================================================================
% GENERAL RESPONSE TO EDITOR
% =============================================================================
\noindent Dear Editor,

\vspace{0.5em}

We sincerely thank you and the reviewers for the thorough and
constructive evaluation of our manuscript. The comments have been
invaluable in improving the quality and clarity of our work.

We have carefully addressed every comment and suggestion. Below, we
provide a detailed point-by-point response. For convenience:
\begin{itemize}[leftmargin=2em]
    \item \textcolor{ReviewerBlue}{\textit{Reviewer comments are shown
          in blue italics}}
    \item \textbf{Our responses are in black bold}
    \item \textcolor{ChangeRed}{\textbf{Changes to the manuscript are
          described in red}}
\end{itemize}

All changes in the revised manuscript are highlighted in
\textcolor{ChangeRed}{red} for easy identification.

\vspace{1em}

% =============================================================================
% RESPONSE TO EDITOR
% =============================================================================
\section*{Response to Editor}

\EditorComment{%
    The reviewers have raised several concerns that need to be addressed
    before the manuscript can be considered for publication. Please pay
    particular attention to the experimental evaluation and the clarity
    of the methodology section.
}

\Response{%
    We thank the editor for summarizing the key areas for improvement.
    We have substantially revised the experimental section (now
    Section~5) with additional baselines and datasets, and rewritten
    the methodology section (Section~3) for improved clarity. A
    detailed accounting of all changes follows below.
}

\vspace{2em}

% =============================================================================
% REVIEWER 1
% =============================================================================
\section*{Response to Reviewer 1}
\ResetPoints

\noindent We thank Reviewer 1 for the careful reading and insightful
comments. We address each point below.

% --- Example: Accepting a suggestion and making changes ---
\ReviewerPoint{%
    The motivation for using topological features is not sufficiently
    clear. Why should topological features be preferred over standard
    deep learning features? The authors should provide a more
    compelling justification.
}

\Response{%
    We thank the reviewer for this important observation. We agree that
    the motivation needed strengthening. Topological features capture
    global structural properties (such as connected components, loops,
    and voids) that are invariant to continuous deformations --- a
    property that standard convolutional or fully-connected features
    do not inherently possess. We have expanded the Introduction
    (Section~1, paragraphs 2--3) with a concrete example demonstrating
    how topological features capture information missed by standard
    feature extractors, supported by a new illustrative figure
    (Fig.~2).
}

\Change{%
    Revised Section~1, paragraphs 2--3 (page 2). Added new Fig.~2
    showing a comparison of feature representations. Added references
    [X, Y, Z] to support the motivation.
}

% --- Example: Partially accepting a suggestion ---
\ReviewerPoint{%
    The experimental evaluation is limited. The authors only test on
    two datasets. I recommend adding experiments on at least two more
    datasets, including a large-scale benchmark such as ImageNet.
}

\Response{%
    We appreciate this suggestion and agree that broader evaluation
    strengthens the paper. We have added experiments on two additional
    datasets: \textbf{CIFAR-100} and \textbf{Tiny-ImageNet} (a
    manageable subset of ImageNet with 200 classes).

    Regarding full ImageNet: We respectfully note that our method
    involves computing persistent homology, which has $O(n^3)$
    complexity in the number of simplices. Running this on
    full-resolution ImageNet (1.2M images) would require computational
    resources beyond our current capacity. However, the Tiny-ImageNet
    results (Table~3) demonstrate scalability to a meaningful scale.
    We have added a discussion of computational constraints and
    scalability in Section~5.4.
}

\Change{%
    Added Section~5.2 with CIFAR-100 results (Table~2).
    Added Section~5.3 with Tiny-ImageNet results (Table~3).
    Added Section~5.4 discussing computational complexity and
    scalability.
}

% --- Example: Respectfully disagreeing ---
\ReviewerPoint{%
    The comparison with Method X [Reference 15] is unfair because the
    authors use different preprocessing for their method versus the
    baseline. Both should use identical preprocessing pipelines.
}

\Response{%
    We appreciate the reviewer raising this concern about fair
    comparison. However, we respectfully clarify that the preprocessing
    steps \textit{are} identical for all methods. Specifically, all
    methods receive the same normalized input images (zero-mean,
    unit-variance, resized to $224 \times 224$). The topological
    feature extraction in our method occurs \textit{after} this shared
    preprocessing step, as an additional feature computation module
    (see Fig.~1, Step~2). This is analogous to how different methods
    may compute different internal representations from the same
    preprocessed input.

    To make this clearer, we have added an explicit statement in
    Section~4.2 confirming that all methods share identical
    preprocessing, and we have added a preprocessing column to
    Table~1.
}

\Change{%
    Clarified preprocessing in Section~4.2 (page 7, paragraph 1).
    Added ``Preprocessing'' column to Table~1 showing identical
    settings for all methods.
}

% --- Example: Accepting a minor comment ---
\ReviewerPoint{%
    There are several typographical errors throughout the paper.
    For example, ``topoligical'' on page 3, and inconsistent
    capitalization of ``Persistent Homology'' versus ``persistent
    homology.''
}

\Response{%
    We thank the reviewer for catching these errors. We have
    carefully proofread the entire manuscript and corrected all
    typographical errors. We now consistently use lowercase
    ``persistent homology'' throughout, except at the beginning of
    sentences.
}

\Change{%
    Corrected all typographical errors throughout the manuscript.
    Standardized terminology and capitalization.
}

\vspace{2em}

% =============================================================================
% REVIEWER 2
% =============================================================================
\section*{Response to Reviewer 2}
\ResetPoints

\noindent We thank Reviewer 2 for the constructive feedback and
thoughtful suggestions. We address each point below.

% --- Example response ---
\ReviewerPoint{%
    The paper lacks a proper ablation study. It is unclear which
    components of the proposed method contribute most to the
    performance gains. I strongly recommend adding an ablation
    experiment.
}

\Response{%
    We completely agree with this suggestion. We have added a
    comprehensive ablation study (Section~5.5, Table~4) that
    systematically evaluates the contribution of each component:
    %
    \begin{enumerate}[label=(\alph*), leftmargin=2em]
        \item Full model: 94.3\% accuracy
        \item Without topological features: 92.1\% ($-$2.2\%)
        \item Without regularization: 93.5\% ($-$0.8\%)
        \item Without data augmentation: 93.0\% ($-$1.3\%)
    \end{enumerate}
    %
    The results confirm that topological features provide the largest
    individual contribution. We have also added a discussion
    interpreting these findings.
}

\Change{%
    Added new Section~5.5 ``Ablation Study'' with Table~4 (page 9).
    Added discussion of ablation findings (page 9, paragraphs 2--3).
}

\ReviewerPoint{%
    The related work section does not discuss [Important Paper by
    Author et al., 2023], which is highly relevant to this work.
}

\Response{%
    We thank the reviewer for pointing us to this important reference.
    We have added a discussion of [Author et al., 2023] in Section~2.2
    and explain how our approach differs: while [Author et al.] focus
    on [their approach], our method addresses [different aspect] by
    [key difference].
}

\Change{%
    Added discussion of [Author et al., 2023] in Section~2.2
    (page 4, paragraph 3). Added citation [XX] to the bibliography.
}

\ReviewerPoint{%
    The writing quality needs improvement. Several sentences are
    overly long and difficult to follow, particularly in Section 3.
}

\Response{%
    We have thoroughly revised Section~3 for clarity, breaking long
    sentences into shorter ones, adding transitional phrases, and
    restructuring paragraphs for better flow. We have also had the
    revised manuscript proofread by a native English speaker.
}

\Change{%
    Substantially revised Section~3 for improved readability.
    Revised writing throughout the manuscript.
}

\vspace{2em}

% =============================================================================
% REVIEWER 3 (if applicable, copy the structure above)
% =============================================================================
% \section*{Response to Reviewer 3}
% \ResetPoints
% ...

% =============================================================================
% SUMMARY OF CHANGES
% =============================================================================
\section*{Summary of All Changes}

For convenience, we list all major changes to the revised manuscript:

\begin{enumerate}[leftmargin=2em, itemsep=0.4em]
    \item \textcolor{ChangeRed}{Expanded motivation in Section~1 with
          new illustrative example (Fig.~2)}
    \item \textcolor{ChangeRed}{Added discussion of [Author et al.,
          2023] to Section~2.2}
    \item \textcolor{ChangeRed}{Substantially revised Section~3 for
          clarity and readability}
    \item \textcolor{ChangeRed}{Clarified preprocessing details in
          Section~4.2}
    \item \textcolor{ChangeRed}{Added experiments on CIFAR-100
          (Section~5.2, Table~2) and Tiny-ImageNet (Section~5.3,
          Table~3)}
    \item \textcolor{ChangeRed}{Added computational analysis in
          Section~5.4}
    \item \textcolor{ChangeRed}{Added ablation study in Section~5.5,
          Table~4}
    \item \textcolor{ChangeRed}{Corrected all typographical errors and
          standardized terminology}
    \item \textcolor{ChangeRed}{Added [N] new references to the
          bibliography}
\end{enumerate}

\vspace{2em}

\noindent We hope that the revised manuscript addresses all concerns
satisfactorily. We are grateful for the opportunity to improve our
work and welcome any further suggestions.

\vspace{1em}

\noindent Sincerely,\\
\textbf{[Corresponding Author Name]}\\
On behalf of all co-authors

\end{document}
